\documentclass[12pt]{article}
\usepackage[brazilian]{babel}
\usepackage[utf8]{inputenc}
\usepackage{setspace}
\usepackage{boxedminipage}
\usepackage{amsmath}
\usepackage{latexsym}
\usepackage{multirow}
\usepackage[pdftex]{graphicx}
\usepackage{float}
\usepackage{url}
\usepackage{tikz}
\usetikzlibrary{bayesnet}
\usepackage{blkarray}

\renewcommand{\familydefault}{\sfdefault}
\newcommand{\question}[2] {\vspace{0.3in}\noindent{\subsection*{Exercício #1. #2} \vspace{0.15in}}}
\renewcommand{\part}[1] {{\vspace{0.15in}\noindent\textbf (#1)} \vspace{0.10in}}
\newcommand{\answer}[1]{{\fontfamily{\rmdefault}\selectfont \textbf{R:} #1}}
\newcommand{\overbar}[1]{\mkern 2mu\overline{\mkern-2mu#1\mkern-2mu}\mkern 2mu}

\setlength{\paperheight}{29.7cm}
\setlength{\textheight}{23.0cm}
\setlength{\textwidth}{16.5cm}
\setlength{\oddsidemargin}{0.0cm}
\setlength{\topmargin}{-1.0cm}
\pagestyle{empty}


\begin{document}
\title{Lista de exercícios de Introdução à Redes Booleanas Probabilisticas}
\author{\large Nome}
\date{\today}
\maketitle

\question{1}{Dada a rede booleana abaixo:}
\begin{center}
\begin{tikzpicture}
  \tikzstyle{gene} = [circle, minimum width=8pt, draw, inner sep=0pt]
  % Define nodes
  \node[gene, label={[yshift=0.1cm] $x_1$}] (x1) {};
  \node[gene, right=2cm of x1, label={[yshift=.1cm] $x_2$}] (x2) {};
  \node[gene, below=2cm of x2, label={[yshift=-1cm] $x_4$}] (x4) {};
  \node[gene, below=2cm of x1, label={[yshift=-1cm] $x_3$}] (x3) {};

  \edge [->, shorten >=.1cm] {x1}{x2};
  \edge [-Bar, shorten >=.1cm] {x4}{x2};
  \edge [->, shorten >=.1cm] {x3}{x4};
  \edge [-Bar, shorten >=.1cm] {x3}{x1};
  \edge [->, shorten >= .1cm] {x2}{x3}
 \end{tikzpicture}
\end{center}

\part{1} Monte a matriz de interação.

\answer{
}

\part{2} Para cada gene, encontre sua expressão booleana.

\answer {
}



\question{2}{Monte a tabela de probabilidade condicional para a rede do
exercício 1 usando o modelo de PBNs de $\alpha$s e $\beta$s.}

\answer {
}



\question{3}{Mostre a tabela de transição de estados para a PBN do 
último exercício.}

\answer {
}

\question{4}{Faça um programa que recebe $n > 0$, $\alpha$, $\beta$ e a
matriz de que representa a rede e devolva a matriz de transição.}

\question{5}{Faça um programa que recebe $n > 0$, uma probabilidade de 
inversão de bits $p$ e a matriz de que representa a rede e devolva a
matriz de transição.}

\question{6}{Faça um programa que receba a matriz de transição e devolva
a matriz estacionária.}

\question{7}{Faça um programa que receba a matriz de transição e devolva
a matriz de probabilidades de fluxo.}

\question{8}{Faça um programa que receba $n > 0$, $\alpha$, $\beta$ e 
a matriz que representa a rede e devolva a matriz de fluxo total.}

\question{9}{Reproduza os resultados do paper "Generating 
Boolean networks with a prescribed attractor structure".}

\question{10}{Completar a tabela com os tratamentos}
\end{document}

