\documentclass[12pt]{article}
\usepackage[brazilian]{babel}
\usepackage[utf8]{inputenc}
\usepackage{setspace}
\usepackage{boxedminipage}
\usepackage{amsmath}
\usepackage{latexsym}
\usepackage{multirow}
\usepackage[pdftex]{graphicx}
\usepackage{float}
\usepackage{url}
\usepackage{tikz}
\usetikzlibrary{bayesnet}
\usepackage{blkarray}

\renewcommand{\familydefault}{\sfdefault}
\newcommand{\question}[2] {\vspace{0.3in}\noindent{\subsection*{Exercício #1. #2} \vspace{0.15in}}}
\renewcommand{\part}[1] {{\vspace{0.15in}\noindent\textbf (#1)} \vspace{0.10in}}
\newcommand{\answer}[1]{{\fontfamily{\rmdefault}\selectfont \textbf{R:} #1}}
\newcommand{\overbar}[1]{\mkern 2mu\overline{\mkern-2mu#1\mkern-2mu}\mkern 2mu}

%\setlength{\parskip}{0.1cm}
\setlength{\paperheight}{29.7cm}
\setlength{\textheight}{23.0cm}
\setlength{\textwidth}{16.5cm}
\setlength{\oddsidemargin}{0.0cm}
\setlength{\topmargin}{-1.0cm}
\pagestyle{empty}


\begin{document}
\title{Lista de exercícios de Introdução à Redes Booleanas Probabilisticas}
\author{\large Gustavo Estrela de Matos}
\date{\today}
\maketitle


\question{2}{Monte a tabela de probabilidade condicional para a rede do
exercício 1 usando o modelo de PBNs de $\alpha$s e $\beta$s}
\answer {

Para $x_1$:

$\begin{array}{cc | c | c}
    x_1 (t) &  x_3 (t) &  P (x_1 (t + 1) = 0 | x_1 (t), x_3 (t)) 
                       &  P (x_1 (t + 1) = 1 | x_1 (t), x_3 (t))\\
    \hline
    X     &     1    &     
                          \frac{e^{\beta}}{e^{\beta} + e^{-\beta}} &
                          \frac{e^{-\beta}}{e^{\beta} + e^{-\beta}} \\
    0     &     0    &     
                          \frac{1}{1 + e^{-\alpha}} & 
                          \frac{e^{-\alpha}}{1 + e^{-\alpha}} \\
    1     &     0    &    
                          \frac{e^{-\alpha}}{1 + e^{-\alpha}} & 
                          \frac{1}{1 + e^{-\alpha}} \\
\end{array}$

\bigbreak
Para $x_2$:

$\begin{array}{ccc | c | c}
    x_2 (t) &  x_1 (t) & x_4 (t) 
                       & P (x_2 (t + 1) = 0 | x_1 (t), x_2 (t), x_4 (t)) 
                       & P (x_2 (t + 1) = 1 | x_1 (t), x_2 (t), x_4 (t))\\
    \hline
    X     &     1      & 0 
                       & \frac{e^{-\beta}}{e^{\beta} + e^{-\beta}} 
                       & \frac{e^{\beta}}{e^{\beta} + e^{-\beta}} \\
    X     &     0      & 1  
                       & \frac{e^{\beta}}{e^{\beta} + e^{-\beta}} 
                       & \frac{e^{-\beta}}{e^{\beta} + e^{-\beta}} \\
    0     &     0      & 0
                       & \frac{1}{1 + e^{-\alpha}} 
                       & \frac{e^{-\alpha}}{1 + e^{-\alpha}} \\
    1     &     0      & 0
                       & \frac{e^{-\alpha}}{1 + e^{-\alpha}} 
                       & \frac{1}{1 + e^{-\alpha}} \\
    0     &     1      & 1
                       & \frac{1}{1 + e^{-\alpha}} 
                       & \frac{e^{-\alpha}}{1 + e^{-\alpha}} \\
    1     &     1      & 1
                       & \frac{e^{-\alpha}}{1 + e^{-\alpha}} 
                       & \frac{1}{1 + e^{-\alpha}} \\
\end{array}$

Para $x_3$:

$\begin{array}{cc | c | c}
    x_3 (t) &  x_2 (t) &  P (x_3 (t + 1) = 0 | x_2 (t), x_3 (t)) 
                       &  P (x_3 (t + 1) = 1 | x_2 (t), x_3 (t))\\
    \hline
    X     &     1    &     
                          \frac{e^{-\beta}}{e^{\beta} + e^{-\beta}} &
                          \frac{e^{\beta}}{e^{\beta} + e^{-\beta}} \\
    0     &     0    &     
                          \frac{1}{1 + e^{-\alpha}} & 
                          \frac{e^{-\alpha}}{1 + e^{-\alpha}} \\
    1     &     0    &    
                          \frac{e^{-\alpha}}{1 + e^{-\alpha}} & 
                          \frac{1}{1 + e^{-\alpha}} \\
\end{array}$

\bigbreak

Para $x_4$:

$\begin{array}{cc | c | c}
    x_4 (t) &  x_3 (t) &  P (x_4 (t + 1) = 0 | x_3 (t), x_4 (t)) 
                       &  P (x_4 (t + 1) = 1 | x_3 (t), x_4 (t))\\
    \hline
    X     &     1    &     
                          \frac{e^{-\beta}}{e^{\beta} + e^{-\beta}} &
                          \frac{e^{\beta}}{e^{\beta} + e^{-\beta}} \\
    0     &     0    &     
                          \frac{1}{1 + e^{-\alpha}} & 
                          \frac{e^{-\alpha}}{1 + e^{-\alpha}} \\
    1     &     0    &    
                          \frac{e^{-\alpha}}{1 + e^{-\alpha}} & 
                          \frac{1}{1 + e^{-\alpha}} \\
\end{array}$
}
\end{document}

